\subsection{Knihovna Dlib}
Knihovna Dlib je vyvíjena primárně Davidem Kingem, který je jejím autorem a její počátky tkví již v roce 2002. Tato knihovna je otevřená, multiplatformní a navržena designově na zakázku a komponenty jsou založeny na softwarovém inženýrství, jedná se o sbírku nezávislých softwarových komponent z nichž je každá doprovázena důkladnou dokumentaci a mnoha příklady použítí.

Knihovna se stále rozrůstá hlavně díky dobrovolným přispěvovatelům a konkrétně nyní obsahuje celou řadu užitečných nástrojů. Například softwarové komponenty pro práci se sítí, vlákna, grafické rozhraní, komplexní datové struktury, lineární algebra, statistické strojové učení, zpracování obrazu, data mining, XML a parsování textu, numerická optimalizace, Bayeské sítě. V uplynulých letech byla velká část vývoje zaměřena na širokou sadu nástrojů statického strojového učení, avšak knihovna zůstává univerzální.  

Jádrem filozofie této knihovny je věnování snadnému používání a přenositelnosti. Proto je kód navržen tak, aby nebylo po uživateli vyžadováno cokoli ručně konfigurovat nebo instalovat. K dosažení tohoto cíle je veškerý kód specifický a pro konkrétní platformu omezen a obalen pomocí API rozhraní. Všechno ostatní je buď navrstveno na těchto obalech nebo napsané v normě ISO standard C++. V současné době je známo, že knihovna pracuje na systémech OS X, MS Windows, Linux, Solaris, BSD, HP-UX. Knihovna by měla také pracovat na libovolné platformě POSIX, ale není otestovaná na všech dostupných verzích. 
Z této knihovny byl v práci použit pouze FHOG detektor objektů.
