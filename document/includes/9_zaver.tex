\section{Závěr}
Dle zadání práce byly otestované vybrané metodiky z knihoven OpenCV a Dlib pro rozpoznávání chodců. Z těchto metodik byly vybrány algoritmy z OpenCV - Histogram orientovaných gradientů, kaskádové klasifikátory a z Dlib knihovny byl vybrán Histogram orientovaných gradientů a tyto algoritmy byly implementovány v aplikaci. Obě tyto knihovny disponují nástroji pro natrénování klasifikátoru, které se dají poté použít na samotnou detekci.

Trénování klasifikátoru hraje důležitou roli pro samotnou detekci, protože zvyšuje přesnost detekce a účinnost klasifikátoru.  Další významnou roli na samotnou úspěšnost klasifikátoru je množina trénovacích dat a nastavení trénování klasifikátoru. Tato data by měla být co nejpřesnější a obsahovat minimum stínu a minimum artefaktů. Jak bylo již v tomto dokumentu zmíněno, aplikace obsahuje i nástroje pro otestování klasifikátoru na daných testovacíh datech, které slouží především k přibližnému zvolení parametrů klasifikátoru.

Tato aplikace by mohla být nadále rozšiřována dalšími klasifikátory pro porovnání rychlosti a výsledkům k již naimplementovaným. 