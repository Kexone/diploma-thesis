\section{Úvod}
Detekce chodců je velmi náročný úkol, který v~posledních letech přitahuje velkou pozornost. Velkou zásluhu má na tom i~Navneet Dalal a~Bill Triggs ve své práci \cite{hog:dalal}, kde pomocí metodiky histogramu orientovaných gradientů úspěšně detekují chodce v~obrazech. Tato práce se z~větší části inspiruje tímto dokumentem a~rozšiřuje jej o~substrakci pozadí, která má za úkol urychlit a~zlepšit samotnou detekci chodců v~obrazech.  Principem tohoto algoritmu je spustit tuto detekci pouze v~oblastech obrazu, který není statický, a~kdy se tedy nemusí jednat o~chodce. Ovšem tento typ algoritmu lze použít pouze na videosnímcích, kde můžeme aplikovat metodu substrakce pozadí.

Aplikace tohoto typu mohou mít široké uplatnění jak v~osobním, tak v~industriálním využití. Může se jednat o~bezpečnostní prvky, například na letištích, kde program může sledovat pohyb daného chodce a~vyhodnocovat tak jeho chování, jakým je kamera průmyslového vozidla, kde řidič může přehlédnout chodce v~blízkosti vozidla, a~předejít tak neštěstí, kdy program může zamezit pohybu vozidla. Dalšími příklady mohou být detekce chodců na přechodech pro chodce, v~továrnách, kde se může pomocí rozšíření algoritmu o~rozpoznání obrazu vyhodnocovat chování a~docházka zaměstnanců. 

Nutno podotknout, že detekce chodců v~obrazech není pro lidi tak obtížným ůkolem a~dokážou bez námahy rozpoznat všechny osoby v~obraze. Pro stroje je, ale tento úkol velkou výzvou. Chodci v~obrazech mohou mít různý tvar těla, barvu kůže, jiný postoj, také různý počet vrstev a~barev oblečení na sobě. Mohou být také z~části zakrytí nějakým objektem v~obraze, který nepodléhá samotné detekci, a~tedy vyhodnoceny záporně. Také zde má velký vliv vzdálenost osoby v~obraze, kdy se pro strojovou doménu může stát nedetekovatelný. 

Hlavním cílem této práce bylo optimalizovat a~urychlit algoritmus pro počítače s~architekturou ARM a~využít tak jejich značný výkon. 

Experimenty této práce se zaměří jak na trénování a~testování klasifikátoru s~různými parametry, tak na detekci chodců na samostatných snímcích, ale také i~ve videosekvencích, které budou detailněji zaznamenány a~vyhodnoceny. 

V~druhé kapitole budou představeny hlavní výzvy a~problémy detekce chodců v~obrazech.

Jedna z~kapitol se bude věnovat metodikám pro detekci chodců a~využitých knihoven pro zpracování obrazu v~této práci. 

V~dalších kapitolách budou uvedena a~popsána zařízení, na kterých byl program otestován, popis funkcí, kterými tato aplikace disponuje, a~v~neposlední řadě experimenty a~jejich dosažené výsledky. 