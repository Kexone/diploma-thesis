% Zadame pozadovane vstupy pro generovani titulnich stran.
\ThesisAuthor{Jakub Ševčík}

\CzechThesisTitle{Detekce chodců na embedded systémech}

\EnglishThesisTitle{Pedestrian Detection on Embedded Systems}

\SubmissionDate{30. dubna 2018}

% Pokud nechceme nikomu dekovat makro zapoznamkujeme.
\Thanks{Rád bych na tomto místě poděkoval všem, kteří mi s prací pomohli, protože bez nich by tato práce nevznikla.}

% Zadame cestu a jmeno souboru ci nekolika souboru s digitalizovanou podobou zadani prace.
% Pokud toto makro zapoznamkujeme sazi se stranka s upozornenim.
\ThesisAssignmentImagePath{figures/assignment}

% Zadame soubor s digitalizovanou podobou prohlaseni autora zaverecne prace.
% Pokud toto makro zapoznamkujeme sazi se cisty text prohlaseni.
%\AuthorDeclarationImageFile{figures/AuthorDeclaration.jpg}

%\ThesisAccessRestriction{Zde vložte text dohodnutého omezení přístupu k Vaší práci, chránící například firemní know-how. Zde vložte text dohodnutého omezení přístupu k Vaší práce, chránící například firemní know-how. A zavazujete se, že:
%\begin{enumerate}
%\item podle \textsection{} 5 o práci nikomu neřeknete,
%\item po obhajobě na ni zapomenete a
%\item budete popírat její existenci.
%\end{enumerate}
%A ještě jeden důležitý odstavec. A ještě jeden důležitý odstavec.
%A ještě jeden důležitý odstavec. A ještě jeden důležitý odstavec.
%A ještě jeden důležitý odstavec. A ještě jeden důležitý odstavec.
%Konec textu dohodnutého omezení přístupu k Vaší práci.}

% Zadame soubor s digitalizovanou podobou souhlasu spolupracujici prav. nebo fyz. osoby.
% Pokud toto makro zapoznamkujeme sazi se cisty text souhlasu.
%\CooperatingPersonsDeclarationImageFile{Figures/CoopPersonDeclaration.jpg}

\CzechAbstract{Cílem práce je prozkoumat a otestovat dosavadní techniky rozpoznávání chodců v obrazech.  Primárním cílem práce je otestovat a pokusit se optimalizovat tyto rozpoznávácí algoritmy na vybraných embedded zařízeních. K rozpoznávání byly použity knihovny OpenCV a Dlib. Použité algoritmy jsou popsány v textu. Součástí práce je také srovnání úspěšnosti a rychlosti jednotlivých detektorů. }

\CzechKeywords{detekce chodců, embbeded systémy, optimalizace}

\EnglishAbstract{The goal of work is to explore and test the existing techniques of pedestrian detection in scenes. The primary objective of the thesis is to test and attempt to optimize these detection algorithms on selected embedded devices. For detection were used the OpenCV and Dlib libraries. The algorithms used are described in the text. Part of the thesis is also a comparison of the successful and speed of individual detectors.}

\EnglishKeywords{Pedestrian Detection, Embedded Systems, optimalization}

\AddAcronym{IoT}{Internet of Things - Internet věcí}
\AddAcronym{C++}{Programovací jazyk C Plus Plus}
\AddAcronym{XML}{Extensible Markup Language}
\AddAcronym{YML}{YAML Ain't Markup Language}
\AddAcronym{ARM}{Advanced RISC Machine - architektura počítačů s nízkou elektrickou spotřebou energie}
\AddAcronym{RGB}{Red Green Blue - barevný prostor, složený ze tří barevných kanálů}
\AddAcronym{SVM}{Support vector machines}
\AddAcronym{API}{Application Programming Interface - rozhraní pro programování aplikací}
\AddAcronym{Dlib}{Otevřená cross-platform knihovna pro zpracování obrazu a strojového učení}
\AddAcronym{OpenCV}{Otevřená knihovna pro zpracování obrazu a strojového učení}
\AddAcronym{SIMD}{Single Instruction Multiple Data - typ počítačové architektury}
\AddAcronym{NEON}{Pokročilé rozšíření architektury SIMD pro procesory ARM Cortex-A a Cortex-R52 }

