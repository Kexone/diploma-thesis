\section{Dataset} % @TODO 
V průběhu let bylo shromážděno nespočet trénovacích sad pro chodce, které jsou veřejné na internetu. Všechny mají rozdílné charakteristiky, slabé a silné stránky. 

Sada INRIA \cite{} patří mezi nejstarší sady vůbec. Ačkoliv obsahuje poměrně málo vzorků, přináší díky tomu vysokou kvalitu anotací chodců v různých prostředích, což je většinou hlavní důvod, proč je tato sada vybrána pro trénování. Na rozdíl sady ETH \cite{} a TUD-Brussels \cite{} patří mezi středně velké video vzorky. Další známou trénovací sadou je Daimler \cite{}, kterou nelze použít pro všechny metodiky, poněvadž poskytuje vzorku v odstínech šedi. Sady ETH, KITTI \cite{} a Daimler-Stereo \cite{} poskytují stereofonní informace. Na obrázku \ref{fig:daimler_stereo} je ukázka tréninkové a testovací sady Daimler-Stereo. 

\begin{figure}[H]
\centering
\includegraphics[width=16cm]{figures/daimler_stereo}
\caption{Ukázka trénovací a testovací sady Daimler-Stereo \cite{}}
\label{fig:daimler_stereo}
\end{figure}

V dnešní době převládají Caltech-USA \cite{} a KITTI jako trénovací sady pro chodce. Obě poskytují poměrně velký počet vzorků. Caltech-USA vyniká počtem možným využitím a obsahuje více než 2300 jedinečných anotací chodců. KITTI zase předčí svou testovací sadou, která je více rozmanitější, ale na druhou stranu se tato sada nepoužívá tak často. V roce 2014 vznikl projekt PETA (Pedestrian Attribute Recognition At Far Distance), tento projekt kombinuje mnoho trénovacích sad a vytváří tak jednu velkou a rozmanitou sadu, která usnadnuje učení robustních detektorů s dobrých výkonem.

V této práci používám převážně trénovací sadu CUHK01 \cite{}, sadu pana Sudip Das \cite{sudipdas} a jako negativní sadu Daimler \cite{} a vzorky vytvořené z fotografií.

